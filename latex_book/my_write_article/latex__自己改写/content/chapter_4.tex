

\chapter{代码块}
%%%%%%%%%%%%%%%%%%%%%%%%
\section{代码块}



插入代码时,用到包\imp{listings},并在 settings 中code.txe 中设置代码块格式

代码如下:
\begin{lstlisting}[language={Tex}]
	
	% 代码段
	\begin{figure}
		\centering
		\def\svgscale{0.5}
		%\input{figure_one.pdf_tex} 
		% 当与main.tex 位于同一层文件夹中
		\import{figures}{figuretwo.pdf_tex}
		 % 当与main.tex 不位于同一层文件夹中
		\caption{插入SVG 图片}
	\end{figure}

\end{lstlisting}

c++代码如下:
\begin{lstlisting}[language={[ANSI]C}]
	#include<stdio.h>
	int main()
	{
		const int a =10;
		return 0;
	}

\end{lstlisting}



代码文件存在 code 文件夹中。在code文件夹下code.tex 中设置自定义代码格式\\
代码如下:\\
\begin{lstlisting}[language={Tex}]
	
\lstdefinestyle{CPP_STYLE}{% CPP_STYLE 是格式的名字 ,定义代码格式
    belowcaptionskip=1\baselineskip,
    breaklines=true,
    frame=none,
    xleftmargin=\parindent,
    language=C,
    showstringspaces=false,
    basicstyle=\footnotesize\ttfamily,
    keywordstyle=\color[RGB]{40,40,255},
    commentstyle=\it\color[RGB]{0,96,96}, 
    identifierstyle=\color{blue},
    stringstyle=\rmfamily\slshape\color[RGB]{128,0,0}, 
    tabsize=2,
}

\end{lstlisting}

使用时如下:\\
\begin{lstlisting}[language={Tex}]
    %\lstinputlisting[style=CPP_STYLE]{code/one.cpp}
    %代码的文件名,源代码要utf-8,不然有中文的注释那些是乱码
\end{lstlisting}
文件代码:

\lstinputlisting[style=CPP_STYLE]{code/one.cpp}%代码的文件名,源代码要utf-8,不然有中文的注释那些是乱码
