
\chapter{Hyperlinks and references}
\label{href}

%%%%%%%%%%%%%%%%%%%%%%%%%%%%%%%%%%%%%%%%%%%%%%%%%%%%%%%%%%%
%%%%%%%%%%%%%%%%%%%%%%%%%%%%%%%%%%%%%%%%%%%%%%%%%%%%%%%%%%%

\section{The package hyperref}

%The package \imp{hyperref} is the package for referring to labeled elements of a document and hyperlinks. Now, chapters, sections, equations, figures, tables and other elements can be labeled and referred to, e.g., \autoref{math_fe}, \autoref{math_msymb} and \autoref{href}. These are clickable links which in the pdf redirects the reader to the referred element (with ALT+LEFT you can then go back to where you were reading). Here, different alternatives can be used, e.g., \ref{href}, \autoref{href} or \hyperref[href]{Chapter \ref*{href}}. Depending on which language you have to write something, you may need language options (e.g., ngerman for German hyperlinks).

\imp{hyperref} 是用于引用标记过的元素和超链接的包。
现在,章节、章节、方程、图, 现在,章节,部分,方程,图表格和其他元素可以被标记和引用,
例如 \autoref{math_fe}、\autoref{math_msymb} 和 \autoref{href}。
表格和其他元素可以被标记和引用,例如\autoref{math_fe}、\autoref{math_msymb}和\autoref{href}。
这些是可点击的, pdf的读者可以链接到这些元素。(用Alt+LEFT 你可以重新回到你刚才阅读的地方, 这要取决与你用的pdf阅读软件的快捷键。)。
这里使用了不同的链接方式,来达到不同的显示效果, 但是作用是一样的,都是链接。
比如:\ref{href}, \autoref{href} or \hyperref[href]{第\ref*{href}章}。由于你使用的语言,你可能需要设置语言选项。

现在我要显示公式的编号,但是不做链接。\autoref*{math_fe}

%%%%%%%%%%%%%%%%%%%%%%%%%%%%%%%%%%%%%%%%%%%%%%%%%%%%%%%%%%%
%%%%%%%%%%%%%%%%%%%%%%%%%%%%%%%%%%%%%%%%%%%%%%%%%%%%%%%%%%%

\section{超链接、邮件和电脑软件}

超链接可以以这两种方式添加,比如:\url{http://miktex.org/} or \href{http://miktex.org/}{点击这里}。
发送邮件可以这样添加:\href{mailto:name.lastname@address.org}{name.lastname@address.org}。
还可以通过以下方式运行一些视频或软件,\href{run:attachments/video.mp4}{video}, 
\href{run:attachments/3Dpractice.ggb}{geogebra}

%%%%%%%%%%%%%%%%%%%%%%%%%%%%%%%%%%%%%%%%%%%%%%%%%%%%%%%%%%%
%%%%%%%%%%%%%%%%%%%%%%%%%%%%%%%%%%%%%%%%%%%%%%%%%%%%%%%%%%%

\section{参考文献}

Bibtex 文档可以使用手工创建,或者使用bib管理软件,e.g,  
\href{http://www.mendeley.com/}{Mendeley} or \href{http://citavi.com/en/index.html}{Citavi}. 
The bibtex 文档要用 \imp{bibliography} 指明文件位置, 用 \imp{bibliographystyle} 设置样式,
和一个包来让你使用这些功能。使用 \imp{cite/p} 文档内的元素就可以被引用, e.g., \cite{Hill1952} and \citep{Kroner1977}. 
在编译的时候,一定要使用bibtex (see compiling options). 要是bibligoraphy的样式没有改变,
你可能需要删除\textbf{除了main.tex以外}的电脑生成的其他文件, 然后重新编译。记住,是除了\textbf{main.tex以外}, 电脑编译
过程中自动生成的那些文件。

%%%%%%%%%%%%%%%%%%%%%%%%%%%%%%%%%%%%%%%%%%%%%%%%%%%%%%%%%%%
%%%%%%%%%%%%%%%%%%%%%%%%%%%%%%%%%%%%%%%%%%%%%%%%%%%%%%%%%%%
%%%%%%%%%%%%%%%%%%%%%%%%%%%%%%%%%%%%%%%%%%%%%%%%%%%%%%%%%%%
