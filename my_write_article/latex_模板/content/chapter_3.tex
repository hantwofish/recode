
\chapter{数学}

%%%%%%%%%%%%%%%%%%%%%%%%%%%%%%%%%%%%%%%%%%%%%%%%%%%%%%%%%%%
%%%%%%%%%%%%%%%%%%%%%%%%%%%%%%%%%%%%%%%%%%%%%%%%%%%%%%%%%%%

\section{Equations and math mode}

虽然我会介绍数学公式的写法,但是现在网上已经有很多数学公式编辑器,
你需要的是一款好的,牛逼的编辑器。这里给大家介绍一款\href{https://www.latexlive.com/}{\LaTeX 公式编辑器}

创造一个自动编号的数学等式。
\begin{equation}
	f(x)
	= A^{23}_{ijkl}(x) \int_0^l\limits g(y,x) \frac{\partial h(y,x)}{\partial y} d y \ .
	\label{math_fe}
\end{equation}

这里是在同一行使用的数学模式,$x^2$
$$
	x^3
$$

\[ 
	x*4
\]
首先我们需要\imp{amsmath}这个包,加载这个包以后,就可以使用equation来写公式。
引用被标记过的使用\eqref{math_fe}。
怎么听上去像飞雷神之术?方便记忆,amsmath的ams是American Mathematical Society的缩写,
而eqref就是equation reference的缩写。一般{LaTeX}的这些写法都是英语或者英语的缩写。

不需要编号的等式,可以使用\imp{equation*}来创建。

\begin{equation*}
	f(x)
	= A^{23}_{ijkl}(x) \int_0^l\limits g(y,x) \frac{\partial h(y,x)}{\partial y} d y \ ,
\end{equation*}
or alternatively 
\[
	f(x)
	= A^{23}_{ijkl}(x) \int_0^l\limits g(y,x) \frac{\partial h(y,x)}{\partial y} d y \ .
\]

你可以使用equation array(等式数组),来写多个等式。等式数组就是几个等式
的江湖黑话。
\begin{eqnarray}
	f(x)
		&=& (x+a)^2 \label{eqna1} \\
		&=& (x+a)(x+a) \label{eqna2}\\
		&=& x^2 + 2 x a + a^2
	\label{eqnaend}
\end{eqnarray}

分别使用 \eqref{eqna1} 和 \eqref{eqnaend} 来实现文中引用标签标记过的等式. 
同样,不需要数字编号就在eqnarray后面加一个*.
\begin{eqnarray*}
	f(x)
		&=& (x+a)^2 \\
		&=& (x+a)(x+a) \\
		&=& x^2 + 2 x a + a^2
\end{eqnarray*}

还可以用align的方式, align比较灵活,它可以完成上述所有的公式。

\begin{align*}
	f(x)
		&= (x+a)^2 \\
		&= (x+a)(x+a) \\
		&= x^2 + 2 x a + a^2
\end{align*}

\begin{align}
	f(x)
	= A^{23}_{ijkl}(x) \int_0^l\limits g(y,x) \frac{\partial h(y,x)}{\partial y} d y \ ,
\end{align}

要想数学公式等式和文本显示在一起,而不是独自显示,就使用 \imp{\$\$} e.g., $f(x) = x^{234}_{ijkl}$.

%%%%%%%%%%%%%%%%%%%%%%%%%%%%%%%%%%%%%%%%%%%%%%%%%%%%%%%%%%%
%%%%%%%%%%%%%%%%%%%%%%%%%%%%%%%%%%%%%%%%%%%%%%%%%%%%%%%%%%%

\section{数组环境和矩阵}

数组环境可以用来创建网格对齐的数学元素, array环境必须用在数学的环境里面。e.g.,
\begin{equation}
	\begin{array}{rcr}
	x + y + z 
		& m_{1234567} 
		& 13425436543634 \\
	A^{23}_{ijkl}(x) \int_0^l\limits g(y,x) \frac{\partial h(y,x)}{\partial y} d y 
		& n_{k} 
		& 123
	\end{array}
\end{equation}

几个等式也可以这样排列。
\begin{equation}
	\begin{array}{rcl}
	f(x)
	&=& A^{23}_{ijkl}(x) \int_0^l\limits g(y,x) \frac{\partial h(y,x)}{\partial y} d y \\
	&=& 7 x \ .
	\end{array}
\end{equation}

这个也可以用align环境来写,如下:
\begin{align*}
	f(x)
	&= A^{23}_{ijkl}(x) \int_0^l\limits g(y,x) \frac{\partial h(y,x)}{\partial y} d y \\
	&= 7 x \ .
\end{align*}
使用 \imp{eqnarray} 环境,有些时候显示的公式不太令人满意。
为了充分显示公式可以用命令\imp{displaystyle}。 
\begin{equation}
	\begin{array}{rcl}
	f(x)
	&=& \displaystyle A^{23}_{ijkl}(x) \int_0^l\limits g(y,x) \frac{\partial h(y,x)}{\partial y} d y \\
	&=& 7 x \ .
	\end{array}
\end{equation} 

数组环境也可以用来显示矩阵, e.g., 
\begin{equation}
	\left(
	\begin{array}{cccc}
	123123 & 324 & 214 & 4 \\
	43& 345345645 & 45353465 & 346
	\end{array}
	\right) \ .
\end{equation}

你也可以用如下的环境来创建矩阵,实际上,我推荐这个方式。

\begin{equation}
	\begin{pmatrix}
	123123 & 324 & 214 & 4 \\
	43& 345345645 & 45353465 & 346
	\end{pmatrix} 
	\quad
	\begin{bmatrix}
	123123 & 324 & 214 & 4 \\
	43& 345345645 & 45353465 & 346
	\end{bmatrix}
\end{equation}

\begin{equation}
    \begin{Bmatrix}
        1 & 4 \\
        2 & 5 \\
        3 & 6
    \end{Bmatrix}
\end{equation}
%%%%%%%%%%%%%%%%%%%%%%%%%%%%%%%%%%%%%%%%%%%%%%%%%%%%%%%%%%%
%%%%%%%%%%%%%%%%%%%%%%%%%%%%%%%%%%%%%%%%%%%%%%%%%%%%%%%%%%%

\section{数学字体}

数学字体需要导言区使用\imp{amssymb} 包. 下面是一些具体的字体。
\[
	\begin{array}{lccccc}
	\text{default} & r & R & Sym^+ & \gamma & \Gamma \\
	\text{bb} & \mathbb{r} & \mathbb{R} & \mathbb{Sym^+} & \mathbb{\gamma}& \mathbb{\Gamma} \\ 
	\text{bf} & \mathbf{r} & \mathbf{R} & \mathbf{Sym^+} & \mathbf{\gamma}& \mathbf{\Gamma} \\
	\text{cal} & \mathcal{r} & \mathcal{R} & \mathcal{Sym^+} & \mathcal{\gamma}& \mathcal{\Gamma} \\
	\text{frak} & \mathfrak{r} & \mathfrak{R} & \mathfrak{Sym^+} & \mathfrak{\gamma}& \mathfrak{\Gamma} \\
	\text{it} & \mathit{r} & \mathit{R} & \mathit{Sym^+} & \mathit{\gamma}& \mathit{\Gamma} \\
	\text{rm} & \mathrm{r} & \mathrm{R} & \mathrm{Sym^+} & \mathrm{\gamma}& \mathrm{\Gamma} \\
	\text{sf} & \mathsf{r} & \mathsf{R} & \mathsf{Sym^+} & \mathsf{\gamma}& \mathsf{\Gamma} \\
	\text{tt} & \mathtt{r} & \mathtt{R} & \mathtt{Sym^+} & \mathtt{\gamma}& \mathtt{\Gamma} \\
	\text{boldsymbol} & \boldsymbol{r} & \boldsymbol{R} & \boldsymbol{Sym^+} & \boldsymbol{\gamma}& \boldsymbol{\Gamma} \\
	\end{array}
\]
\imp{mathbb}命令在不同的包下有不同的效果, e.g., \imp{euscript} and \imp{lucida} (look for latex math fonts in stackexchange). 
自己定义一些宏会减少你的工作量, e.g., $\bbA$.

%%%%%%%%%%%%%%%%%%%%%%%%%%%%%%%%%%%%%%%%%%%%%%%%%%%%%%%%%%%
%%%%%%%%%%%%%%%%%%%%%%%%%%%%%%%%%%%%%%%%%%%%%%%%%%%%%%%%%%%

\section{Math symbols}
\label{math_msymb}

数学符号太多了。下面这些是其中之一, e.g., 
\begin{equation}
	\sum , \int , \iiint , \nabla , \cdot , \times , \otimes , \rightarrow , \Rightarrow , \bigcup , \in , \subset .
\end{equation}

记得我前面介绍的网址,用不着背这些。


%%%%%%%%%%%%%%%%%%%%%%%%%%%%%%%%%%%%%%%%%%%%%%%%%%%%%%%%%%%
%%%%%%%%%%%%%%%%%%%%%%%%%%%%%%%%%%%%%%%%%%%%%%%%%%%%%%%%%%%
%%%%%%%%%%%%%%%%%%%%%%%%%%%%%%%%%%%%%%%%%%%%%%%%%%%%%%%%%%%
